% Options for packages loaded elsewhere
\PassOptionsToPackage{unicode}{hyperref}
\PassOptionsToPackage{hyphens}{url}
%
\documentclass[
]{article}
\usepackage{amsmath,amssymb}
\usepackage{lmodern}
\usepackage{ifxetex,ifluatex}
\ifnum 0\ifxetex 1\fi\ifluatex 1\fi=0 % if pdftex
  \usepackage[T1]{fontenc}
  \usepackage[utf8]{inputenc}
  \usepackage{textcomp} % provide euro and other symbols
\else % if luatex or xetex
  \usepackage{unicode-math}
  \defaultfontfeatures{Scale=MatchLowercase}
  \defaultfontfeatures[\rmfamily]{Ligatures=TeX,Scale=1}
\fi
% Use upquote if available, for straight quotes in verbatim environments
\IfFileExists{upquote.sty}{\usepackage{upquote}}{}
\IfFileExists{microtype.sty}{% use microtype if available
  \usepackage[]{microtype}
  \UseMicrotypeSet[protrusion]{basicmath} % disable protrusion for tt fonts
}{}
\makeatletter
\@ifundefined{KOMAClassName}{% if non-KOMA class
  \IfFileExists{parskip.sty}{%
    \usepackage{parskip}
  }{% else
    \setlength{\parindent}{0pt}
    \setlength{\parskip}{6pt plus 2pt minus 1pt}}
}{% if KOMA class
  \KOMAoptions{parskip=half}}
\makeatother
\usepackage{xcolor}
\IfFileExists{xurl.sty}{\usepackage{xurl}}{} % add URL line breaks if available
\IfFileExists{bookmark.sty}{\usepackage{bookmark}}{\usepackage{hyperref}}
\hypersetup{
  pdftitle={Basic Queries},
  hidelinks,
  pdfcreator={LaTeX via pandoc}}
\urlstyle{same} % disable monospaced font for URLs
\usepackage[margin=1in]{geometry}
\usepackage{color}
\usepackage{fancyvrb}
\newcommand{\VerbBar}{|}
\newcommand{\VERB}{\Verb[commandchars=\\\{\}]}
\DefineVerbatimEnvironment{Highlighting}{Verbatim}{commandchars=\\\{\}}
% Add ',fontsize=\small' for more characters per line
\usepackage{framed}
\definecolor{shadecolor}{RGB}{248,248,248}
\newenvironment{Shaded}{\begin{snugshade}}{\end{snugshade}}
\newcommand{\AlertTok}[1]{\textcolor[rgb]{0.94,0.16,0.16}{#1}}
\newcommand{\AnnotationTok}[1]{\textcolor[rgb]{0.56,0.35,0.01}{\textbf{\textit{#1}}}}
\newcommand{\AttributeTok}[1]{\textcolor[rgb]{0.77,0.63,0.00}{#1}}
\newcommand{\BaseNTok}[1]{\textcolor[rgb]{0.00,0.00,0.81}{#1}}
\newcommand{\BuiltInTok}[1]{#1}
\newcommand{\CharTok}[1]{\textcolor[rgb]{0.31,0.60,0.02}{#1}}
\newcommand{\CommentTok}[1]{\textcolor[rgb]{0.56,0.35,0.01}{\textit{#1}}}
\newcommand{\CommentVarTok}[1]{\textcolor[rgb]{0.56,0.35,0.01}{\textbf{\textit{#1}}}}
\newcommand{\ConstantTok}[1]{\textcolor[rgb]{0.00,0.00,0.00}{#1}}
\newcommand{\ControlFlowTok}[1]{\textcolor[rgb]{0.13,0.29,0.53}{\textbf{#1}}}
\newcommand{\DataTypeTok}[1]{\textcolor[rgb]{0.13,0.29,0.53}{#1}}
\newcommand{\DecValTok}[1]{\textcolor[rgb]{0.00,0.00,0.81}{#1}}
\newcommand{\DocumentationTok}[1]{\textcolor[rgb]{0.56,0.35,0.01}{\textbf{\textit{#1}}}}
\newcommand{\ErrorTok}[1]{\textcolor[rgb]{0.64,0.00,0.00}{\textbf{#1}}}
\newcommand{\ExtensionTok}[1]{#1}
\newcommand{\FloatTok}[1]{\textcolor[rgb]{0.00,0.00,0.81}{#1}}
\newcommand{\FunctionTok}[1]{\textcolor[rgb]{0.00,0.00,0.00}{#1}}
\newcommand{\ImportTok}[1]{#1}
\newcommand{\InformationTok}[1]{\textcolor[rgb]{0.56,0.35,0.01}{\textbf{\textit{#1}}}}
\newcommand{\KeywordTok}[1]{\textcolor[rgb]{0.13,0.29,0.53}{\textbf{#1}}}
\newcommand{\NormalTok}[1]{#1}
\newcommand{\OperatorTok}[1]{\textcolor[rgb]{0.81,0.36,0.00}{\textbf{#1}}}
\newcommand{\OtherTok}[1]{\textcolor[rgb]{0.56,0.35,0.01}{#1}}
\newcommand{\PreprocessorTok}[1]{\textcolor[rgb]{0.56,0.35,0.01}{\textit{#1}}}
\newcommand{\RegionMarkerTok}[1]{#1}
\newcommand{\SpecialCharTok}[1]{\textcolor[rgb]{0.00,0.00,0.00}{#1}}
\newcommand{\SpecialStringTok}[1]{\textcolor[rgb]{0.31,0.60,0.02}{#1}}
\newcommand{\StringTok}[1]{\textcolor[rgb]{0.31,0.60,0.02}{#1}}
\newcommand{\VariableTok}[1]{\textcolor[rgb]{0.00,0.00,0.00}{#1}}
\newcommand{\VerbatimStringTok}[1]{\textcolor[rgb]{0.31,0.60,0.02}{#1}}
\newcommand{\WarningTok}[1]{\textcolor[rgb]{0.56,0.35,0.01}{\textbf{\textit{#1}}}}
\usepackage{graphicx}
\makeatletter
\def\maxwidth{\ifdim\Gin@nat@width>\linewidth\linewidth\else\Gin@nat@width\fi}
\def\maxheight{\ifdim\Gin@nat@height>\textheight\textheight\else\Gin@nat@height\fi}
\makeatother
% Scale images if necessary, so that they will not overflow the page
% margins by default, and it is still possible to overwrite the defaults
% using explicit options in \includegraphics[width, height, ...]{}
\setkeys{Gin}{width=\maxwidth,height=\maxheight,keepaspectratio}
% Set default figure placement to htbp
\makeatletter
\def\fps@figure{htbp}
\makeatother
\setlength{\emergencystretch}{3em} % prevent overfull lines
\providecommand{\tightlist}{%
  \setlength{\itemsep}{0pt}\setlength{\parskip}{0pt}}
\setcounter{secnumdepth}{-\maxdimen} % remove section numbering
\ifluatex
  \usepackage{selnolig}  % disable illegal ligatures
\fi

\title{Basic Queries}
\author{}
\date{\vspace{-2.5em}}

\begin{document}
\maketitle

\hypertarget{libraries}{%
\subsubsection{Libraries}\label{libraries}}

\begin{Shaded}
\begin{Highlighting}[]
\FunctionTok{library}\NormalTok{(RMySQL)}
\end{Highlighting}
\end{Shaded}

\begin{verbatim}
## Warning: le package 'RMySQL' a été compilé avec la version R 4.1.3
\end{verbatim}

\begin{verbatim}
## Le chargement a nécessité le package : DBI
\end{verbatim}

\begin{verbatim}
## Warning: le package 'DBI' a été compilé avec la version R 4.1.1
\end{verbatim}

\begin{Shaded}
\begin{Highlighting}[]
\FunctionTok{library}\NormalTok{(bslib)}
\end{Highlighting}
\end{Shaded}

\begin{verbatim}
## Warning: le package 'bslib' a été compilé avec la version R 4.1.3
\end{verbatim}

\begin{verbatim}
## 
## Attachement du package : 'bslib'
\end{verbatim}

\begin{verbatim}
## L'objet suivant est masqué depuis 'package:utils':
## 
##     page
\end{verbatim}

1. From the following table : stade, write a SQL query to count the
number of venues for EURO cup 2016. Return number of venues.

\begin{Shaded}
\begin{Highlighting}[]
\FunctionTok{dbGetQuery}\NormalTok{(df, }\StringTok{"SELECT count(venue\_name) as \textquotesingle{}number of venue\textquotesingle{}}
\StringTok{                FROM stade"}\NormalTok{)}
\end{Highlighting}
\end{Shaded}

\begin{verbatim}
##   number of venue
## 1              10
\end{verbatim}

2. From the following table : player\_mast, write a SQL query to count
the number of countries that participated in the 2016-EURO Cup

\begin{Shaded}
\begin{Highlighting}[]
\FunctionTok{dbGetQuery}\NormalTok{(df, }\StringTok{"SELECT count(distinct(team\_id)) as \textquotesingle{}number of countries\textquotesingle{}}
\StringTok{                FROM player\_mast"}\NormalTok{)}
\end{Highlighting}
\end{Shaded}

\begin{verbatim}
##   number of countries
## 1                  24
\end{verbatim}

3. From the following table : goal\_details, write a SQL query to find
the number of goals scored within normal play during the EURO cup 2016

\begin{Shaded}
\begin{Highlighting}[]
\FunctionTok{dbGetQuery}\NormalTok{(df, }\StringTok{"SELECT count(goal\_id) as \textquotesingle{}goals scored within normal play\textquotesingle{}}
\StringTok{                FROM goal\_details"}\NormalTok{)}
\end{Highlighting}
\end{Shaded}

\begin{verbatim}
##   goals scored within normal play
## 1                             108
\end{verbatim}

4. From the following table : match\_mast, write a SQL query to find the
number of matches that ended with a result.

\begin{Shaded}
\begin{Highlighting}[]
\FunctionTok{dbGetQuery}\NormalTok{(df, }\StringTok{"SELECT count(results) as \textquotesingle{}matches\textquotesingle{}}
\StringTok{                FROM match\_mast}
\StringTok{                WHERE results = \textquotesingle{}WIN\textquotesingle{}"}\NormalTok{)}
\end{Highlighting}
\end{Shaded}

\begin{verbatim}
##   matches
## 1      40
\end{verbatim}

5. From the following table : match\_mast, write a SQL query to find the
number of matches that ended in draws.

\begin{Shaded}
\begin{Highlighting}[]
\FunctionTok{dbGetQuery}\NormalTok{(df, }\StringTok{"SELECT count(results) as \textquotesingle{}matches end in draws\textquotesingle{}}
\StringTok{                FROM match\_mast}
\StringTok{                WHERE results = \textquotesingle{}DRAW\textquotesingle{}"}\NormalTok{)}
\end{Highlighting}
\end{Shaded}

\begin{verbatim}
##   matches end in draws
## 1                   11
\end{verbatim}

6. From the following table : match\_mast, write a SQL query to find out
when the Football EURO cup 2016 will begin.

\begin{Shaded}
\begin{Highlighting}[]
\FunctionTok{dbGetQuery}\NormalTok{(df, }\StringTok{"SELECT play\_date as \textquotesingle{}begin date\textquotesingle{}}
\StringTok{                FROM match\_mast}
\StringTok{                WHERE match\_no = 1"}\NormalTok{)}
\end{Highlighting}
\end{Shaded}

\begin{verbatim}
##   begin date
## 1 2016-06-11
\end{verbatim}

7. From the following table : goal\_details, write a SQL query to find
the number of self-goals scored during the 2016 European Championship.

\begin{Shaded}
\begin{Highlighting}[]
\FunctionTok{dbGetQuery}\NormalTok{(df, }\StringTok{"SELECT count(*) as \textquotesingle{}self{-}goals scored\textquotesingle{}}
\StringTok{                FROM goal\_details}
\StringTok{                WHERE goal\_type = \textquotesingle{}O\textquotesingle{}"}\NormalTok{)}
\end{Highlighting}
\end{Shaded}

\begin{verbatim}
##   self-goals scored
## 1                 3
\end{verbatim}

8. From the following table : match\_mast, write a SQL query to count
the number of matches ended with a win results in-group stage.

\begin{Shaded}
\begin{Highlighting}[]
\FunctionTok{dbGetQuery}\NormalTok{(df, }\StringTok{"SELECT count(*) as \textquotesingle{}win matches\textquotesingle{}}
\StringTok{                FROM match\_mast}
\StringTok{                WHERE play\_stage = \textquotesingle{}G\textquotesingle{} AND results = \textquotesingle{}WIN\textquotesingle{}"}\NormalTok{)}
\end{Highlighting}
\end{Shaded}

\begin{verbatim}
##   win matches
## 1          25
\end{verbatim}

9. From the following table : penalty\_shootout, write a SQL query to
find the number of matches that resulted in a penalty shootout

\begin{Shaded}
\begin{Highlighting}[]
\FunctionTok{dbGetQuery}\NormalTok{(df, }\StringTok{"SELECT count(distinct(match\_no)) as \textquotesingle{}matches that resulted in a penalty shootout\textquotesingle{}}
\StringTok{                FROM penalty\_shootout"}\NormalTok{)}
\end{Highlighting}
\end{Shaded}

\begin{verbatim}
##   matches that resulted in a penalty shootout
## 1                                           3
\end{verbatim}

10. From the following table : match\_mast, write a SQL query to find
number of matches decided by penalties in the Round 16.

\begin{Shaded}
\begin{Highlighting}[]
\FunctionTok{dbGetQuery}\NormalTok{(df, }\StringTok{"SELECT count(distinct(match\_no)) as \textquotesingle{}matches decided by penalties in the Round 16\textquotesingle{}}
\StringTok{                FROM match\_mast}
\StringTok{                WHERE play\_stage = \textquotesingle{}R\textquotesingle{} AND decided\_by = \textquotesingle{}P\textquotesingle{}"}\NormalTok{)}
\end{Highlighting}
\end{Shaded}

\begin{verbatim}
##   matches decided by penalties in the Round 16
## 1                                            1
\end{verbatim}

11. From the following table : goal\_details, write a SQL query to find
the number of goals scored in every match within a normal play schedule.
Sort the result-set on match number. Return match number, number of goal
scored.

\begin{Shaded}
\begin{Highlighting}[]
\FunctionTok{dbGetQuery}\NormalTok{(df, }\StringTok{"SELECT match\_no, }
\StringTok{                       count(goal\_id) as \textquotesingle{}number of goal scored\textquotesingle{}}
\StringTok{                FROM goal\_details}
\StringTok{                GROUP BY match\_no}
\StringTok{                ORDER BY match\_no"}\NormalTok{)}
\end{Highlighting}
\end{Shaded}

\begin{verbatim}
##    match_no number of goal scored
## 1         1                     3
## 2         2                     1
## 3         3                     3
## 4         4                     2
## 5         5                     1
## 6         6                     1
## 7         7                     2
## 8         8                     1
## 9         9                     2
## 10       10                     2
## 11       11                     2
## 12       12                     2
## 13       13                     3
## 14       14                     2
## 15       15                     2
## 16       16                     3
## 17       17                     2
## 18       19                     1
## 19       20                     4
## 20       21                     3
## 21       22                     3
## 22       23                     2
## 23       25                     1
## 24       27                     3
## 25       29                     1
## 26       30                     1
## 27       31                     2
## 28       32                     3
## 29       33                     3
## 30       34                     6
## 31       35                     1
## 32       36                     1
## 33       37                     2
## 34       38                     1
## 35       39                     1
## 36       40                     3
## 37       41                     3
## 38       42                     4
## 39       43                     2
## 40       44                     3
## 41       45                     2
## 42       46                     4
## 43       47                     2
## 44       48                     7
## 45       49                     2
## 46       50                     2
## 47       51                     1
\end{verbatim}

12. From the following table : match\_mast, write a SQL query to find
the matches in which no stoppage time was added during the first half of
play. Return match no, date of play, and goal scored.

\begin{Shaded}
\begin{Highlighting}[]
\FunctionTok{dbGetQuery}\NormalTok{(df, }\StringTok{"SELECT match\_no,}
\StringTok{                       play\_date,}
\StringTok{                       goal\_score}
\StringTok{                FROM match\_mast}
\StringTok{                WHERE stop1\_sec = 0"}\NormalTok{)}
\end{Highlighting}
\end{Shaded}

\begin{verbatim}
##   match_no  play_date goal_score
## 1        4 2016-06-12        1-1
\end{verbatim}

13. From the following table : match\_details, write a SQL query to
count the number of matches that ended in a goalless draw at the group
stage. Return number of matches.

\begin{Shaded}
\begin{Highlighting}[]
\FunctionTok{dbGetQuery}\NormalTok{(df, }\StringTok{"SELECT count(distinct(match\_no)) as \textquotesingle{}number of matches that ended in a goalless draw\textquotesingle{} }
\StringTok{                FROM match\_details}
\StringTok{                WHERE play\_stage = \textquotesingle{}G\textquotesingle{} AND win\_lose = \textquotesingle{}D\textquotesingle{} AND goal\_score = 0"}\NormalTok{)}
\end{Highlighting}
\end{Shaded}

\begin{verbatim}
##   number of matches that ended in a goalless draw
## 1                                               4
\end{verbatim}

14. From the following table : match\_details, write a SQL query to
calculate the number of matches that ended in a single goal win,
excluding matches decided by penalty shootouts. Return number of
matches.

\begin{Shaded}
\begin{Highlighting}[]
\FunctionTok{dbGetQuery}\NormalTok{(df, }\StringTok{"SELECT count(distinct(match\_no)) as \textquotesingle{}number of matches that ended in a single goal win\textquotesingle{} }
\StringTok{                FROM match\_details}
\StringTok{                WHERE win\_lose = \textquotesingle{}W\textquotesingle{} AND goal\_score = 1 AND decided\_by != \textquotesingle{}P\textquotesingle{}"}\NormalTok{)}
\end{Highlighting}
\end{Shaded}

\begin{verbatim}
##   number of matches that ended in a single goal win
## 1                                                13
\end{verbatim}

15. From the following table : player\_in\_out, write a SQL query to
count the number of players replaced in the tournament. Return number of
players as ``Player Replaced''.

\begin{Shaded}
\begin{Highlighting}[]
\FunctionTok{dbGetQuery}\NormalTok{(df, }\StringTok{"SELECt count(player\_id) as \textquotesingle{}Player Replaced\textquotesingle{}}
\StringTok{                FROM player\_in\_out}
\StringTok{                WHERE in\_out = \textquotesingle{}I\textquotesingle{}"}\NormalTok{)}
\end{Highlighting}
\end{Shaded}

\begin{verbatim}
##   Player Replaced
## 1             293
\end{verbatim}

16. From the following table : player\_in\_out, write a SQL query to
count the total number of players replaced during normal playtime.
Return number of players as ``Player Replaced''.

\begin{Shaded}
\begin{Highlighting}[]
\FunctionTok{dbGetQuery}\NormalTok{(df, }\StringTok{"SELECt  count(player\_id) as \textquotesingle{}Player Replaced\textquotesingle{}}
\StringTok{                FROM player\_in\_out}
\StringTok{                WHERE in\_out =\textquotesingle{}I\textquotesingle{} AND play\_schedule = \textquotesingle{}NT\textquotesingle{}"}\NormalTok{)}
\end{Highlighting}
\end{Shaded}

\begin{verbatim}
##   Player Replaced
## 1             275
\end{verbatim}

17. From the following table : player\_in\_out, write a SQL query to
count the number of players who were replaced during the stoppage time.
Return number of players as ``Player Replaced''.

\begin{Shaded}
\begin{Highlighting}[]
\FunctionTok{dbGetQuery}\NormalTok{(df, }\StringTok{"SELECt  count(player\_id) as \textquotesingle{}Player Replaced\textquotesingle{}}
\StringTok{                FROM player\_in\_out}
\StringTok{                WHERE in\_out =\textquotesingle{}I\textquotesingle{} AND play\_schedule = \textquotesingle{}ST\textquotesingle{}"}\NormalTok{)}
\end{Highlighting}
\end{Shaded}

\begin{verbatim}
##   Player Replaced
## 1               9
\end{verbatim}

18. From the following table : player\_in\_out, write a SQL query to
count the number of players who were replaced during the first half.
Return number of players as ``Player Replaced''.

\begin{Shaded}
\begin{Highlighting}[]
\FunctionTok{dbGetQuery}\NormalTok{(df, }\StringTok{"SELECt  count(player\_id) as \textquotesingle{}Player Replaced\textquotesingle{}}
\StringTok{                FROM player\_in\_out}
\StringTok{                WHERE in\_out =\textquotesingle{}I\textquotesingle{} AND play\_schedule=\textquotesingle{}NT\textquotesingle{} AND play\_half = \textquotesingle{}1\textquotesingle{}"}\NormalTok{)}
\end{Highlighting}
\end{Shaded}

\begin{verbatim}
##   Player Replaced
## 1               3
\end{verbatim}

19. From the following table : match\_details, write a SQL query to
count the total number of goalless draws played in the entire
tournament. Return number of goalless draws

\begin{Shaded}
\begin{Highlighting}[]
\FunctionTok{dbGetQuery}\NormalTok{(df, }\StringTok{"SELECT count(distinct(match\_no)) as \textquotesingle{}number of matches that ended in a goalless draw in the entire tournament\textquotesingle{} }
\StringTok{                FROM match\_details}
\StringTok{                WHERE win\_lose = \textquotesingle{}D\textquotesingle{} AND goal\_score = 0"}\NormalTok{)}
\end{Highlighting}
\end{Shaded}

\begin{verbatim}
##   number of matches that ended in a goalless draw in the entire tournament
## 1                                                                        4
\end{verbatim}

20. From the following table : player\_in\_out, write a SQL query to
calculate the total number of players who were replaced during the extra
time.

\begin{Shaded}
\begin{Highlighting}[]
\FunctionTok{dbGetQuery}\NormalTok{(df, }\StringTok{"SELECt  count(player\_id) as \textquotesingle{}Player Replaced\textquotesingle{}}
\StringTok{                FROM player\_in\_out}
\StringTok{                WHERE in\_out=\textquotesingle{}I\textquotesingle{} AND play\_schedule=\textquotesingle{}ET\textquotesingle{}"}\NormalTok{)}
\end{Highlighting}
\end{Shaded}

\begin{verbatim}
##   Player Replaced
## 1               9
\end{verbatim}

21. From the following table : player\_in\_out, write a SQL query to
count the number of substitutes during various stages of the tournament.
Sort the result-set in ascending order by play-half, play-schedule and
number of substitute happened. Return play-half, play-schedule, number
of substitute happened.

\begin{Shaded}
\begin{Highlighting}[]
\FunctionTok{dbGetQuery}\NormalTok{(df, }\StringTok{"SELECT count(*),play\_half, play\_schedule}
\StringTok{                FROM player\_in\_out}
\StringTok{                WHERE in\_out =\textquotesingle{}I\textquotesingle{}}
\StringTok{                GROUP BY play\_half, play\_schedule}
\StringTok{                ORDER BY play\_half, play\_schedule, count(*) DESC"}\NormalTok{)}
\end{Highlighting}
\end{Shaded}

\begin{verbatim}
##   count(*) play_half play_schedule
## 1        4         1            ET
## 2        3         1            NT
## 3        5         2            ET
## 4      272         2            NT
## 5        9         2            ST
\end{verbatim}

22. From the following table : penalty\_shootout, write a SQL query to
count the number of shots taken in penalty shootouts matches. Number of
shots as ``Number of Penalty Kicks''.

\begin{Shaded}
\begin{Highlighting}[]
\FunctionTok{dbGetQuery}\NormalTok{(df, }\StringTok{"SELECT count(*) as \textquotesingle{}Number of Penalty Kicks\textquotesingle{}, match\_no}
\StringTok{                FROM penalty\_shootout}
\StringTok{                GROUP BY match\_no"}\NormalTok{)}
\end{Highlighting}
\end{Shaded}

\begin{verbatim}
##   Number of Penalty Kicks match_no
## 1                      10       37
## 2                       9       45
## 3                      18       47
\end{verbatim}

23. From the following table : penalty\_shootout, write a SQL query to
count the number of shots that were scored in penalty shootouts matches.
Return number of shots scored goal as ``Goal Scored by Penalty Kicks''.

\begin{Shaded}
\begin{Highlighting}[]
\FunctionTok{dbGetQuery}\NormalTok{(df, }\StringTok{"SELECT count(*) as \textquotesingle{}Goal Scored by Penalty Kicks\textquotesingle{}}
\StringTok{                FROM penalty\_shootout}
\StringTok{                WHERE score\_goal = \textquotesingle{}Y\textquotesingle{}"}\NormalTok{)}
\end{Highlighting}
\end{Shaded}

\begin{verbatim}
##   Goal Scored by Penalty Kicks
## 1                           28
\end{verbatim}

24. From the following table : penalty\_shootout, write a SQL query to
count the number of shots missed or saved in penalty shootout matches.
Return number of shots missed as ``Goal missed or saved by Penalty
Kicks''.

\begin{Shaded}
\begin{Highlighting}[]
\FunctionTok{dbGetQuery}\NormalTok{(df, }\StringTok{"SELECT count(*) as \textquotesingle{}Goal missed or saved by Penalty Kicks\textquotesingle{}}
\StringTok{                FROM penalty\_shootout}
\StringTok{                WHERE score\_goal = \textquotesingle{}N\textquotesingle{}"}\NormalTok{)}
\end{Highlighting}
\end{Shaded}

\begin{verbatim}
##   Goal missed or saved by Penalty Kicks
## 1                                     9
\end{verbatim}

25. From the following table : player\_booked, write a SQL query to
count the number of bookings in each half of play within the normal play
schedule. Return play\_half, play\_schedule, number of booking happened.

\begin{Shaded}
\begin{Highlighting}[]
\FunctionTok{dbGetQuery}\NormalTok{(df, }\StringTok{"SELECT count(*), play\_half}
\StringTok{                FROM player\_booked}
\StringTok{                WHERE play\_schedule = \textquotesingle{}NT\textquotesingle{}}
\StringTok{                GROUP BY play\_half"}\NormalTok{)}
\end{Highlighting}
\end{Shaded}

\begin{verbatim}
##   count(*) play_half
## 1       61         1
## 2      123         2
\end{verbatim}

26. From the following table : player\_booked, write a SQL query to
count the number of bookings during stoppage time.

\begin{Shaded}
\begin{Highlighting}[]
\FunctionTok{dbGetQuery}\NormalTok{(df, }\StringTok{"SELECT count(*) as \textquotesingle{}number of bookings during stoppage time\textquotesingle{}}
\StringTok{                FROM player\_booked}
\StringTok{                WHERE play\_schedule = \textquotesingle{}ST\textquotesingle{}"}\NormalTok{)}
\end{Highlighting}
\end{Shaded}

\begin{verbatim}
##   number of bookings during stoppage time
## 1                                      10
\end{verbatim}

27. From the following table : player\_booked, write a SQL query to
count the number of bookings that happened in extra time.

\begin{Shaded}
\begin{Highlighting}[]
\FunctionTok{dbGetQuery}\NormalTok{(df, }\StringTok{"SELECT count(*) as \textquotesingle{}number of bookings during extra time\textquotesingle{}}
\StringTok{                FROM player\_booked}
\StringTok{                WHERE play\_schedule = \textquotesingle{}ET\textquotesingle{}"}\NormalTok{)}
\end{Highlighting}
\end{Shaded}

\begin{verbatim}
##   number of bookings during extra time
## 1                                    7
\end{verbatim}

\end{document}
